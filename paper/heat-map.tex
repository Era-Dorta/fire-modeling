%%% template.tex
%%%
%%% This LaTeX source document can be used as the basis for your technical
%%% paper or abstract. Regardless of the length of your document, the commands
%%% are all the same.
%%% 
%%% The "\documentclass" command is the first command in your file. If you want to 
%%% prepare a version of your article with line numbers - a "review" version - 
%%% include the "review" parameter:
%%%    \documentclass[review]{acmsiggraph}
%%%

\documentclass{acmsiggraph}

%%% Title of your article or abstract.

\title{An Inverse Problem Approach for Heat-Map Reconstruction Using Photographs}

\author{Garoe Dorta-Perez$^1$ \qquad Yong-Liang Yang$^1$ \qquad Luca Benedetti$^1$ \qquad Dmitry Kit$^1$ \\$^1$University of Bath}
\pdfauthor{Garoe Dorta-Perez}

%%% Used by the ``review'' variation; the online ID will be printed on 
%%% every page of the content.

\TOGonlineid{45678}

% User-generated keywords.

\keywords{radiosity, global illumination, constant time}

% With the "\setcopyright" command the appropriate rights management text will be added
% to your document.

%\setcopyright{none}
%\setcopyright{acmcopyright}
%\setcopyright{acmlicensed}
\setcopyright{rightsretained}
%\setcopyright{usgov}
%\setcopyright{usgovmixed}
%\setcopyright{cagov}
%\setcopyright{cagovmixed}
%\setcopyright{rightsretained}

% The year of publication in the "\copyrightyear" command.

\copyrightyear{2016}

%%% Conference information, from the completed rights management form.
%%% The "\conferenceinfo" command has two parameters: 
%%%    - conference name
%%%    - conference date and location
%%% The "\isbn" field includes the year and month after the article ISBN.

\conferenceinfo{SIGGRAPH 2016 Posters}{July 24-28, 2016, Anaheim, CA} 
\isbn{978-1-4503-ABCD-E/16/07} 
\doi{http://doi.acm.org/10.1145/9999997.9999999}

\begin{document}

%%% This is the ``teaser'' command, which puts an figure, centered, below 
%%% the title and author information, and above the body of the content.

 \teaser{
   \includegraphics[height=1.5in]{images/sampleteaser}
   \caption{Spring Training 2009, Peoria, AZ.}
 }

\maketitle

\begin{abstract}



\end{abstract}

%
% The code below should be generated by the tool at
% http://dl.acm.org/ccs.cfm
% Please copy and paste the code instead of the example below. 
%
\begin{CCSXML}
<ccs2012>
<concept>
<concept_id>10010147.10010371</concept_id>
<concept_desc>Computing methodologies~Computer graphics</concept_desc>
<concept_significance>500</concept_significance>
</concept>
</ccs2012>
\end{CCSXML}

\ccsdesc[500]{Computing methodologies~Computer graphics}

%
% End generated code
%

% The next three commands are required, and insert the user-generated keywords, 
% The CCS concepts list, and the rights management text.
% Please make sure there is a blank line between each of these three commands.

\keywordlist

\conceptlist

\printcopyright

\section{Introduction}


\begin{table}[ht]
  \centering
  \caption{A simple table.}
  \begin{tabular}{|r|l|}
    \hline
    7C0 & hexadecimal \\
    3700 & octal \\ \cline{2-2}
    11111000000 & binary \\
    \hline \hline
    1984 & decimal \\
    \hline
  \end{tabular}
\end{table}
  

\section{Related work}


\subsection{Fire rendering}

Here talk about Pegoraro et. al.

\subsection{Parameter optimization}

Here talk about Dobashi et. al.

\begin{figure}[ht]
  \centering
  \includegraphics[width=3.0in]{images/ferrari_laferrari}
  \caption{Ferrari LaFerrari. Image courtesy Flickr user ``gfreeman23.''}
  \label{fig:ferrari}
\end{figure}

\section{Problem formulation}

\section{Estimation method}

\subsection{Acceleration of fitness function evaluation}

\section{Results and discussion}

\section{Conclusion and future work}

\section*{Acknowledgements}

\bibliographystyle{acmsiggraph}
\nocite{*}
\bibliography{heat-map}
\end{document}
