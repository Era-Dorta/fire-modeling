%%% template.tex
%%%
%%% This LaTeX source document can be used as the basis for your technical
%%% paper or abstract. Regardless of the length of your document, the commands
%%% are all the same.
%%% 
%%% The "\documentclass" command is the first command in your file. If you want to 
%%% prepare a version of your article with line numbers - a "review" version - 
%%% include the "review" parameter:
%%%    \documentclass[review]{acmsiggraph}
%%%

\documentclass{acmsiggraph}

%%% Title of your article or abstract.

\title{An Inverse Problem Approach for Heat-Map Reconstruction Using Photographs}

\author{Garoe Dorta-Perez$^1$ \qquad Yong-Liang Yang$^1$ \qquad Luca Benedetti$^1$ \qquad Dmitry Kit$^1$ \\$^1$University of Bath}
\pdfauthor{Garoe Dorta-Perez}

%%% Used by the ``review'' variation; the online ID will be printed on 
%%% every page of the content.

\TOGonlineid{45678}

% User-generated keywords.

\keywords{fire, heat, inverse problem}

% With the "\setcopyright" command the appropriate rights management text will be added
% to your document.

%\setcopyright{none}
%\setcopyright{acmcopyright}
%\setcopyright{acmlicensed}
\setcopyright{rightsretained}
%\setcopyright{usgov}
%\setcopyright{usgovmixed}
%\setcopyright{cagov}
%\setcopyright{cagovmixed}
%\setcopyright{rightsretained}

% The year of publication in the "\copyrightyear" command.

\copyrightyear{2016}

%%% Conference information, from the completed rights management form.
%%% The "\conferenceinfo" command has two parameters: 
%%%    - conference name
%%%    - conference date and location
%%% The "\isbn" field includes the year and month after the article ISBN.

\conferenceinfo{SIGGRAPH 2016 Posters}{July 24-28, 2016, Anaheim, CA} 
\isbn{978-1-4503-ABCD-E/16/07} 
\doi{http://doi.acm.org/10.1145/9999997.9999999}

\begin{document}

%%% This is the ``teaser'' command, which puts an figure, centered, below 
%%% the title and author information, and above the body of the content.

 \teaser{
   \includegraphics[height=1.5in]{images/sampleteaser}
   \caption{Examples of our method. The heat maps are estimated from real photographs shown on the top left.}
 }

\maketitle

\begin{abstract}



\end{abstract}

%
% The code below should be generated by the tool at
% http://dl.acm.org/ccs.cfm
% Please copy and paste the code instead of the example below. 
%
\begin{CCSXML}
<ccs2012>
<concept>
<concept_id>10010147.10010371</concept_id>
<concept_desc>Computing methodologies~Computer graphics</concept_desc>
<concept_significance>500</concept_significance>
</concept>
</ccs2012>
\end{CCSXML}

\ccsdesc[500]{Computing methodologies~Computer graphics}

%
% End generated code
%

% The next three commands are required, and insert the user-generated keywords, 
% The CCS concepts list, and the rights management text.
% Please make sure there is a blank line between each of these three commands.

\keywordlist

\conceptlist

\printcopyright

\section{Introduction}



\section{Related work}


Many methods have been proposed to render participating; generally approximate solutions for the Radiative Transport Equation (RTE)~\cite{Howell:2002} are computed. 
\cite{Rushmeier:1995} presented a method to perform accurate ray casting on sparse measured data.
The fire was modelled as a series of stacked cylindrical rings, where each ring has uniform properties.
%The total radiance at each point is integrated using a Monte Carlo method, summing up the measured irradiances at sample locations.
A technique to animate fire with suspended particles was introduced by~\cite{Feldman:2003}.
Emitted light was computed using Planck's formula of black body radiation their animation of fire with, however their RGB mapping requires a manual adjustment using images of real explosions.
%Direct illumination shadows were computed using deep shadow maps~\cite{Lokovic:2000}, while scattering and illumination by other objects in the scene used the technique proposed by~\cite{Jensen:2002}.
\cite{Nguyen:2002} proposed a ray marching technique using black body radiation as well, scattering in the media and the observer's visual adaptation to the fire are modelled.
An extension was presented by~\cite{Pegoraro:2006}, the authors' model has physically-based absorption, emission and scattering properties.
The spectroscopic characteristics of different fuels are achieved by modelling the transitions of electrons between different energy states in the molecules.
The method allows for non-linear light trajectories  in the medium due to refraction and it includes visual adaptation effects by means of a post-processing mechanism.
\cite{Horvath:2009} proposed a rendering method whose main objective was user-friendliness for artists.
The authors perform simple volume rendering on several fixed camera slices to generate an image.
Black body radiation is used to compute light emission; the images were motion-blurred with a filter based on the velocities in the slices, and heat distortion was added as post-processing filter defined by the user. 

Inverse rendering problem papers

Other uses of GA

Inverse rendering for light scattering + emission?

\subsection{Fire rendering}

Here talk about Pegoraro et. al.

\subsection{Parameter optimization}

Here talk about Dobashi et. al.


\section{Problem formulation}

\section{Estimation method}

\subsection{Acceleration of fitness function evaluation}

\section{Results and discussion}

\section{Conclusion and future work}

\section*{Acknowledgements}

\bibliographystyle{acmsiggraph}
\nocite{*}
\bibliography{heat-map}
\end{document}
