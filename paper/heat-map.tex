%%% template.tex
%%%
%%% This LaTeX source document can be used as the basis for your technical
%%% paper or abstract. Regardless of the length of your document, the commands
%%% are all the same.
%%% 
%%% The "\documentclass" command is the first command in your file. If you want to 
%%% prepare a version of your article with line numbers - a "review" version - 
%%% include the "review" parameter:
%%%    \documentclass[review]{acmsiggraph}
%%%

\documentclass{acmsiggraph}

%%% Title of your article or abstract.

\title{An Inverse Problem Approach for Heat-Map Reconstruction Using Photographs}

\author{Garoe Dorta-Perez$^1$ \qquad Yong-Liang Yang$^1$ \qquad Luca Benedetti$^1$ \qquad Dmitry Kit$^1$ \\$^1$University of Bath}
\pdfauthor{Garoe Dorta-Perez}

%%% Used by the ``review'' variation; the online ID will be printed on 
%%% every page of the content.

\TOGonlineid{45678}

% User-generated keywords.

\keywords{fire, heat, inverse problem}

% With the "\setcopyright" command the appropriate rights management text will be added
% to your document.

%\setcopyright{none}
%\setcopyright{acmcopyright}
%\setcopyright{acmlicensed}
\setcopyright{rightsretained}
%\setcopyright{usgov}
%\setcopyright{usgovmixed}
%\setcopyright{cagov}
%\setcopyright{cagovmixed}
%\setcopyright{rightsretained}

% The year of publication in the "\copyrightyear" command.

\copyrightyear{2016}

%%% Conference information, from the completed rights management form.
%%% The "\conferenceinfo" command has two parameters: 
%%%    - conference name
%%%    - conference date and location
%%% The "\isbn" field includes the year and month after the article ISBN.

\conferenceinfo{SIGGRAPH 2016 Posters}{July 24-28, 2016, Anaheim, CA} 
\isbn{978-1-4503-ABCD-E/16/07} 
\doi{http://doi.acm.org/10.1145/9999997.9999999}

\begin{document}

%%% This is the ``teaser'' command, which puts an figure, centered, below 
%%% the title and author information, and above the body of the content.

 \teaser{
   \includegraphics[height=1.5in]{images/sampleteaser}
   \caption{Examples of our method. The heat maps are estimated from real photographs shown on the top left.}
 }

\maketitle

\begin{abstract}



\end{abstract}

%
% The code below should be generated by the tool at
% http://dl.acm.org/ccs.cfm
% Please copy and paste the code instead of the example below. 
%
\begin{CCSXML}
<ccs2012>
<concept>
<concept_id>10010147.10010371</concept_id>
<concept_desc>Computing methodologies~Computer graphics</concept_desc>
<concept_significance>500</concept_significance>
</concept>
</ccs2012>
\end{CCSXML}

\ccsdesc[500]{Computing methodologies~Computer graphics}

%
% End generated code
%

% The next three commands are required, and insert the user-generated keywords, 
% The CCS concepts list, and the rights management text.
% Please make sure there is a blank line between each of these three commands.

\keywordlist

\conceptlist

\printcopyright

\section{Introduction}

For centuries humans has been fascinated with fire due to its dangerous nature and attractive presence.
Combustion phenomena are prevalent in daily life, candles, camp fires, explosions, car engines, cooking appliances, etc.
Simulating and visualizing fire related processes has many applications, for example they are widely used for visual effects in the film industry, simulated as part of the virtual environment in the computer games industry; or in the engineering community, where modelling engine combustion and fire safety evaluations are frequently demanded.

Examples of computer generated fires in films include, a planet explosion in Star Trek II, where a particle-based technique by~\cite{Reeves:1983} was used; Shrek featured a dragon exhaling fire, where parametric curves were used to drive the flames~\cite{Lamorlette:2002}; or the more recent work of~\cite{Horvath:2009} based on 2D screen projections for the film Harry Potter and the Deathly Hallows.
In these and in many other applications, using real flames is an expensive and hazardous endeavour.

The computer graphics community has intensively researched the fluid behaviour of water and smoke.
Fire can also be modelled as a fluid, however due to its multiphase flow, chemical reactions and radiative heat transport, the techniques used for water or smoke cannot be directly applied to flames.
As a result of the aforementioned complexity and the interdisciplinary nature of the problem, fire simulation is still an open problem in computer graphics.

A great deal of work done in the area has sacrificed complexity for interactiveness, therefore producing simplified models which hope to deceive the observer by exploiting the chaotic behaviour present in fire motion.
Nevertheless, physically-based simulations incorporate the intrinsic processes that occur in a combustion scenario in order to be able to produce realistic results.

Using the current methods involves a great deal of parameter tuning and trial and error for the user.
Moreover, several techniques require difficult to acquire data, such as the flame temperature, which can be obtained by simulations whose computational cost is quite high or from real data with expensive equipment such as infrared cameras.

We propose a new method to estimate fire volumetric temperature given a real photograph of a flame.
Our aim is two fold, firstly to give the user an intuitive tool to choose fire rendering parameters and, secondly to provide a physically-based volumetric heat values using only data from RGB cameras.
Since color histograms are often used in image retrieval techniques, we use them to evaluate the distance between the generated image and the photograph.


\section{Related work}

In this section we present an overview of the technique used for fire rendering and inverse rendering problems.
A more detailed discussion of fire modelling, simulation and rendering is given in~\cite{Huang:2014}.

\subsection{Fire rendering}

Many physically-based methods have been proposed to render participating media realistically; typically, approximate solutions for the Radiative Transport Equation (RTE)~\cite{Howell:2002} are computed.
\cite{Rushmeier:1995} presented a method to perform accurate ray casting on sparse measured data.
The fire was modelled as a series of stacked cylindrical rings, where each ring has uniform properties.
%The total radiance at each point is integrated using a Monte Carlo method, summing up the measured irradiances at sample locations.
A technique to animate fire with suspended particles was introduced by~\cite{Feldman:2003}.
Emitted light was computed using Planck's formula of black body radiation their animation of fire with, however their RGB mapping requires a manual adjustment using images of real explosions.
%Direct illumination shadows were computed using deep shadow maps~\cite{Lokovic:2000}, while scattering and illumination by other objects in the scene used the technique proposed by~\cite{Jensen:2002}.
\cite{Nguyen:2002} proposed a ray marching technique using black body radiation as well, scattering in the media and the observer's visual adaptation to the fire are modelled.
An extension was presented by~\cite{Pegoraro:2006}, the authors' model has physically-based absorption, emission and scattering properties.
The spectroscopic characteristics of different fuels are achieved by modelling the transitions of electrons between different energy states in the molecules.
The method allows for non-linear light trajectories  in the medium due to refraction and it includes visual adaptation effects by means of a post-processing mechanism.
\cite{Horvath:2009} proposed a rendering method whose main objective was user-friendliness for artists.
The authors perform simple volume rendering on several fixed camera slices to generate an image.
Black body radiation is used to compute light emission; the result is motion-blurred with a filter based on the velocities in the slices, and heat distortion was added as post-processing filter defined by the user.

Rendering flames at interactive frame rates has also been explored, this techniques inevitably sacrifice quality for performance.
\cite{Bridault:2006} used a spectrophotometer to capture photometric distributions of candles.
The intensities are stored on a texture and changes in illumination over time are approximated with an attenuation factor proportional to the size of the flame.
\cite{Zhang:2011} used a plane blending technique were a one-dimensional colour texture is used as a transfer function to convert flow attributes to colours and opacities. 
A texture synthesis technique to generate fire animations was introduced by~\cite{Jamriska:2015}.
The method requires a hand made motion field and an alpha mask of the desired result, both of which are used to generate a new sequence using data from an existing video exemplar.

\subsection{Parameter optimization}

\cite{Dobashi:2012} proposed a method to compute rendering parameters for clouds using a real cloud photograph.
The authors' technique is limited to simplified scenes with a single light and the cloud, were the camera and the light position are fixed.
Under this restrictions a set of images can be precomputed to accelerate the optimization.

Inverse rendering problem papers

Other uses of GA

Inverse rendering for light scattering + emission?




\section{Problem formulation}

\section{Estimation method}

\subsection{Acceleration of fitness function evaluation}

\section{Results and discussion}

\section{Conclusion and future work}

\section*{Acknowledgements}

\bibliographystyle{acmsiggraph}
\nocite{*}
\bibliography{heat-map}
\end{document}
