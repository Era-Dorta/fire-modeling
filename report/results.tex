%------------------------------------------------------------------------------
\chapter{Results}
\label{ch:results}

Images everywhere and proper analysis of what the user is seeing, what is failing and why.
Include default Maya fire effect, pictures of fire from the papers mentioned in the previous work section and a historical overview on the improvements in the shaders.
Could also add some graphs with rendering times per light sample, decreased step, and with and without shadows.

In order to provide a wider comparison that is not only limited to academic results, we will also confront our results with work from industry software.
FumeFX~\cite{FumeFX} is a popular tool for modelling volumetric effects, it has been used in several films and video games, e.g. Ghost Rider 2, Thor, Green Lantern, Warhammer Online and Tomb Raider: Underworld.
A frame from a fire scene in Ghost Rider 2 is shown in Figure~\ref{fig:fumefx}.

\begin{figure}[htbp]
	\centering
	\includegraphics[width=\textwidth]{img/fumefx}
	\caption{Fire created with FumeFX~\cite{FumeFX}.}
	\label{fig:fumefx}
\end{figure}

Since \Maya was used as the base platform for the development of our software, an interesting comparison should include \Mayash's fire effect.
A simple scene with plane, a sphere, a cylinder and a light was created, as shown in Figure~\ref{fig:maya_no_fire_mental_ray}.
The result of rendering the scene with \Mayash's software renderer is shown in Figure~\ref{fig:maya_fire}, note how the sphere and cylinder do not experience any change in their illumination with respect to the scene without the fire.
If \MentalRay is used to render the same scene, the image shown in Figure~\ref{fig:maya_fire_mental_ray} is generated.
Although the fire itself looks more realistic, the only alteration in neighbouring objects  is the appearance of erroneous shadows.
Placing on the flame examples provided by \Maya in our test scene produced the result shown in Figure~\ref{fig:result_maya_fire_example}.
More specifically, the 14th frame of the simulation in the file fire fluid examples ``Flame.ma" was used.
Note that a point light had to be placed in the centre of the flame, and the emitted colour had to be matched manually to the fire colour, to be able to produce the desired effect.

\begin{figure}[htpb]
        \centering
        \begin{subfigure}[t]{\textwidth}
                \includegraphics[width=\textwidth]{img/maya_no_fire_mental_ray}
                \caption{Scene without fire in \Maya, rendered with \MentalRay.}
                 \label{fig:maya_no_fire_mental_ray}
        \end{subfigure}    
        \\     
\end{figure}

\begin{figure}[htpb]
		\ContinuedFloat
		\begin{subfigure}[t]{\textwidth}
                \includegraphics[width=\textwidth]{img/maya_fire}
                \caption{\Maya fire effect on a plane, rendered with \Maya software.}
                \label{fig:maya_fire}
        \end{subfigure}%        
\end{figure}

\begin{figure}[htpb]
        \ContinuedFloat
 		\begin{subfigure}[t]{\textwidth}
                \includegraphics[width=\textwidth]{img/maya_fire_mental_ray}
                \caption{\Maya fire effect on a plane, rendered with \MentalRay.}
                \label{fig:maya_fire_mental_ray}
        \end{subfigure}%             
        \caption{Render tests for \Maya default fire effect.}
        \label{fig:maya_fire_scenes}
\end{figure}

\begin{figure}[htbp]
	\centering
	\includegraphics[width=0.8\textwidth, trim={8cm 0 8cm 10cm}, clip]{img/result_maya_fire_example}
	\caption{One of the \Maya fire fluid examples is our test scene.}
	\label{fig:result_maya_fire_example}
\end{figure}

All the images generated with our shader use 256 light samples, the volumes have resolutions of $256 \times 256 \times 256$ voxels and a ray march step size of half a voxel size.
To give some perspective of the evolution of the software, an image from an early stage of the application is shown in Figure~\ref{fig:result_early_stage}.
At that point standard ray marching was used, a white point light source was placed in the centre of the cube, and the colour of the volume was set manually to orange.
The final colour of each pixel was computed using alpha blending with the densities along the direction of the ray.
The shadows form due to an attenuation factor being computed in the shadow rays.

\begin{figure}[htbp]
	\centering
	\includegraphics[width=\textwidth]{img/result_early_stage}
	\caption{Our shader, early test, basic ray marching.}
	\label{fig:result_early_stage}
\end{figure}

For the next step, we assume that our flame is a black body radiator, as shown in Figure~\ref{fig:result_blackbody}.
In each step of the ray marching, the 256 lights scattered throughout the fire are sampled to compute the colour.
Since no fuel is burning, there is no absorption coefficient to be computed, so the alpha blending is still used.

\begin{figure}[htbp]
	\centering
	\includegraphics[width=0.8\textwidth, trim={8cm 0 8cm 10cm}, clip]{img/result_blackbody}
	\caption{Our shader, flame as a black body radiator.}
	\label{fig:result_blackbody}
\end{figure}

If we model the fire with a real fuel, for example propane, the image shown in Figure~\ref{fig:result_propane_shadows} is generated.
At this point we are rendering with the simplified RTE Equation~\ref{eq:rte_simplified}.
Note how the appearance of the flame improves significantly with respect to the previous result.
If we omit the computation of the attenuation factor for the emitted light $L_e$, rendering times are reduced significantly.
The result is shown in Figure~\ref{fig:result_propane}, the error incurred with this simplification s shown in Figures~\ref{fig:result_diff_shadow} and~\ref{fig:result_diff_shadow_x5}, computed as the absolute of the difference between both images.
The bias introduced is recognizable when there is a ground truth image to compare to, yet if only one image were to be shown, it would be challenging to notice the difference. 

\begin{figure}[htbp]
	\centering
	\includegraphics[width=0.8\textwidth, trim={8cm 0 8cm 10cm}, clip]{img/result_propane_shadows}
	\caption{Our shader, propane flame.}
	\label{fig:result_propane_shadows}
\end{figure}

\begin{figure}[htbp]
	\centering
	\includegraphics[width=0.8\textwidth, trim={8cm 0 8cm 10cm}, clip]{img/result_propane}
	\caption{Our shader, propane without emitted light attenuation.}
	\label{fig:result_propane}
\end{figure}

\begin{figure}[htbp]
	\centering
	\includegraphics[width=0.8\textwidth, trim={8cm 0 8cm 10cm}, clip]{img/result_diff_shadow}
	\caption{Error introduced when the attenuation is not computed.}
	\label{fig:result_diff_shadow}
\end{figure}

\begin{figure}[htbp]
	\centering
	\includegraphics[width=0.8\textwidth, trim={8cm 0 8cm 10cm}, clip]{img/result_diff_shadow_x5}
	\caption{Previous error highlighted by a factor of 5.}
	\label{fig:result_diff_shadow_x5}
\end{figure}

More exotic fuels are also supported by our system, images with copper and sulphur as the burning chemicals are shown in Figures~\ref{fig:result_copper} and~\ref{fig:result_sulfur}.
To be able to render this data, the temperature of the flame must be increased, otherwise the emitted radiation would be outside of the visible spectrum.

\begin{figure}[htbp]
	\centering
	\includegraphics[width=0.8\textwidth, trim={8cm 0 8cm 10cm}, clip]{img/result_copper}
	\caption{Our shader, copper flame.}
	\label{fig:result_copper}
\end{figure}

\begin{figure}[htbp]
	\centering
	\includegraphics[width=0.8\textwidth, trim={8cm 0 8cm 10cm}, clip]{img/result_sulfur}
	\caption{Our shader, sulfur flame.}
	\label{fig:result_sulfur}
\end{figure}

The effects of the visual adaptation transformation are shown in Figure~\ref{fig:visual_adaptation_states}.
As expected, when the visual adaptation factor increases, the centre of the flame shifts to a lighter colour.

\begin{figure}[htpb]
        \centering
        \begin{subfigure}[t]{0.2\textwidth}
                \includegraphics[width=\textwidth, trim={14.5cm 2.5cm 16.5cm 10.5cm}, clip]{img/result_propane}
                \caption{0\%.}
        \end{subfigure}%
        ~ 
        \begin{subfigure}[t]{0.2\textwidth}
                \includegraphics[width=\textwidth, trim={14.5cm 2.5cm 16.5cm 10.5cm}, clip]{img/result_propane_0_25_v}
                \caption{25\%.}
        \end{subfigure}%
        ~ 
        \begin{subfigure}[t]{0.2\textwidth}
                \includegraphics[width=\textwidth, trim={14.5cm 2.5cm 16.5cm 10.5cm}, clip]{img/result_propane_0_5_v}
                \caption{50\%.}
        \end{subfigure}%
        ~ 
        \begin{subfigure}[t]{0.2\textwidth}
                \includegraphics[width=\textwidth, trim={14.5cm 2.5cm 16.5cm 10.5cm}, clip]{img/result_propane_0_75_v}
                \caption{75\%.}
        \end{subfigure}%
        ~ 
        \begin{subfigure}[t]{0.2\textwidth}
                \includegraphics[width=\textwidth, trim={14.5cm 2.5cm 16.5cm 10.5cm}, clip]{img/result_propane_1_v}
                \caption{100\%.}
        \end{subfigure}%
        \caption{The effect of visual adaptation, left to right from no visual adaptation to maximum value.}
        \label{fig:visual_adaptation_states}
\end{figure}

In the work of \cite{Pegoraro:2006}, the authors use data from \cite{Uintah} as the base input for their rendering results.
We have ran some of the example simulations provided with the software, and the data was rendered with our shaders, as shown in Figure \textbf{ADD FIGURE, DRAW CONCLUSIONS FROM IT}.    

An interesting phenomena is the visual appearance of flames which are superimposed, as addition of the colours and transparency is not trivial, as shown in Figure~\ref{fig:result_synthetic}.
To better illustrate this effect, an image of similar situation  with real flames captured with our system is shown in 

\begin{figure}[htpb]
        \centering
        \begin{subfigure}[t]{0.2\textwidth}
                \includegraphics[width=\textwidth]{img/result_synthetic}
                \caption{Our shader, two flames superimposed.}
                \label{fig:result_synthetic}
        \end{subfigure}%
        \qquad %add desired spacing between images, e. g. ~, \quad, \qquad, \hfill etc.
          %(or a blank line to force the subfigure onto a new line)
        \begin{subfigure}[t]{0.15\textwidth}
                \includegraphics[width=\textwidth]{img/vox0087}
                \caption{Captured image of real flames.}
                \label{fig:vox0087}
        \end{subfigure}
        \caption{Superimposed fire effects.}
\end{figure}