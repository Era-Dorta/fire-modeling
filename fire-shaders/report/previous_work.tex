%------------------------------------------------------------------------------
\chapter{Previous Work}
\label{ch:previous_work}

In order to display a realistic fire scene in a computer generated world two differentiated stages are needed.
Firstly, the fire dynamics have to be collected, this can either be done through a data capture session or simulated using a fluid solvers.
Secondly, the previously gathered data is to be visualized on the screen using some rendering technique.

\section{Simulation}
\label{sec:simulation}

Since fire is a multiphase fluid phenomena, it is worthwhile to do a short overview on the main fluid simulation techniques.
Standard fluid simulation has been traditionally geared towards water, nevertheless fire solvers are directly based on such methods.

\section{Fire Simulation}
\label{sec:fire_simulation}

\textbf{Particle-based methods} were the first approach to simulate the visual animation of fire.
A number of particles are emitted from certain locations, each particle has a set of attributes such as shape, velocity, color or lifetime.
The first model with particle systems was presented by~\cite{Reeves:1983}, the particles speed and colour were perturbed with a Gaussian noise at each time step, and the colour was subject to an additional linear perturbation on its lifetime.
Two particle systems were used in a hierarchy, one would control fire spread and the other a single explosion effect.
An extension was proposed by~\cite{Perry:1994}, the authors modified the particle system such that each particle shape would be defined by a series of non-overlapping coplanar triangles.
The transparency of would increase towards the outer vertices, thus providing an improved visual effect.


\textbf{Noise-based methods} focus on synthesizing the high fluctuation present in fire procedurally.
The objective is to approximate the turbulence present in fire with an appropriate statistical model.
Using a variation of Perlin noise, ~\cite{Perlin:1985} presented images of a corona of flames.
However, the method is limited to 2D, where the color is a combination of non-linear arbitrary functions.
This work was extended by~\cite{Perlin:1989} to 3D, where they use volumetric rendering to achieve improved results.

\subsection{Geometry skeleton}

\subsection{Data driven}

\subsection{Physically based}

\subsection{High-speed combustions}

\subsection{Other}
	Erosion and sound paper


%\subsection{Smoke Simulation}
%\label{sec:smoke_simulation}


\section{Rendering}
\label{sec:rendering}


\subsection{Raster-Based}
\label{sec:raster_based}


\subsection{Ray-Tracing-Based}
\label{sec:ray_tracing_based}


\cite{Pegoraro:2006} says ...