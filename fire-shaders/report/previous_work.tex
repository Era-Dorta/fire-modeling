%------------------------------------------------------------------------------
\chapter{Previous Work}
\label{ch:previous_work}

In order to display a realistic fire scene in a computer generated world two differentiated stages are needed.
Firstly, the fire dynamics have to be collected, this can either be done through a data capture session or simulated using a fluid solvers.
Secondly, the previously gathered data is to be visualized on the screen using some rendering technique.

\section{Simulation}
\label{sec:simulation}


\subsection{Fluid Simulation}
\label{sec:fluid_simulation}


\subsection{Fire Simulation}
\label{sec:fire_simulation}


\subsection{Smoke Simulation}
\label{sec:smoke_simulation}


\section{Fire Rendering}
\label{sec:fire_rendering}


\subsection{Raster-Based}
\label{sec:raster_based}


\subsection{Ray-Tracing-Based}
\label{sec:ray_tracing_based}


\cite{Pegoraro:2006} says ...