%------------------------------------------------------------------------------
\chapter{Previous Work}
\label{ch:previous_work}

In order to display a realistic fire scene in a computer generated world two differentiated stages are needed.
Firstly, the fire dynamics have to be collected, this can either be done through a data capture session or simulated using a fluid solvers.
Secondly, the previously gathered data is to be visualized on the screen using some rendering technique.
We refer the readers to the more detailed survey on the topic, which has been recently presented by~\cite{Huang:2014}.

\section{Simulation}
\label{sec:simulation}

%\section{Fire Simulation}
%\label{sec:fire_simulation}

\textbf{Particle-based methods} were the first approach to simulate the visual animation of fire.
A number of particles are emitted from certain locations, each particle has a set of attributes such as shape, velocity, color or lifetime.
The first model with particle systems was presented by~\cite{Reeves:1983}, the particles speed and colour were perturbed with a Gaussian noise at each time step, and the colour was subject to an additional linear perturbation on its lifetime.
Two particle systems were used in a hierarchy, one would control fire spread and the other a single explosion effect.
An extension was proposed by~\cite{Perry:1994}, the authors modified the particle system such that each particle shape would be defined by a series of non-overlapping coplanar triangles.
The transparency of would increase towards the outer vertices, thus providing an improved visual effect.


\textbf{Noise-based methods} focus on synthesizing the high fluctuation present in fire procedurally.
The objective is to approximate the turbulence present in fire with an appropriate statistical model.
Using a variation of Perlin noise, ~\cite{Perlin:1985} presented images of a corona of flames.
However, the method is limited to 2D, where the color is a combination of non-linear arbitrary functions.
This work was extended by~\cite{Perlin:1989} to 3D, where they use volumetric rendering to achieve improved results.

\textbf{Geometry skeleton}

\textbf{Data driven}

\textbf{Physically based} simulate the fire combustion processes, including flame propagation or the chemical reactions that convert fuel into gaseous products.  
Incompressible flow equations were used by~\cite{Stam:1995} to drive a fire simulation.
Given initial fuel conditions, the fire spread is advected on a grid using an advection-diffusion type equation.
Building on the work on a semi-Lagrangian fluid solver of~\cite{Stam:1999}, a model which includes gaseous fuel and gaseous byproducts was proposed by~\cite{Nguyen:2002}.
In order to include the characteristics of the noise-based methods, \cite{Hong:2007} combined the previous model with a set of third-order equations from detonation shock dynamics presented by~\cite{Yao:1996}.
As with the noise-based methods, this addition is visually attractive, yet it is not physically based. 
Capitalizing on the recent advances in GPUs parallel processing power, \cite{Horvath:2009} proposed a fixed camera model.
Particle properties are computed on a three-dimensional coarse grid, which are then projected into several view dependant two-dimensional slices.
The authors' model is based on the assumption that fine variations, which are perpendicular to the projection plane, are not individually visible and, they do not affect significantly the overall flow.
 
\textbf{Other effects} directly related to fire have also been explored.
\cite{Feldman:2003} presented a model to simulate suspended particles during explosions.
An incompressible fluid model drives the motion of air and hot gases, and the suspended particles follow the their movements.
Sound is a important factor to increase the believability of a finished fire animation.
\cite{Chadwick:2011} proposed a method to automatically generate plausible noise given for a given fire simulation.
Low frequency sound is estimated using a physical model whose inputs are the flame front and heat release.
A data driven sound synthesis approach, based on the work by~\cite{Wei:2000}, is applied to generate the high frequency content.

\textbf{Erosion}???


%\subsection{Smoke Simulation}
%\label{sec:smoke_simulation}


\section{Rendering}
\label{sec:rendering}


\subsection{Raster-Based}
\label{sec:raster_based}


\subsection{Ray-Tracing-Based}
\label{sec:ray_tracing_based}


\cite{Pegoraro:2006} says ...