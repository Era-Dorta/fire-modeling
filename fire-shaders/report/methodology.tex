%------------------------------------------------------------------------------
\chapter{Methodology}
\label{ch:methodology}

Not only explain in more detail every equation, what it means, comparison with other papers and ideally improvements or were it fails at least.
Should I explain what is and how to do ray marching? What is the spectrum? How to integrate it to RGB coefficients?
Add all the values for the constants here or in implementation details??

\section{Radiative Transport Equation}
\label{sec:radiative_transport_equation}

The Radiative Transport Equation (RTE) models the variation of spectral radiance in the medium $L(\lxo)$, where $\lambda$ is a given wavelength in $m$, $\x$ is the point of interest in space, and $\omegam$ is a vector that points towards the viewing direction.
The RTE is defined as

\begin{equation}
\begin{split}
(\omega \nabla) L(\lxo) = &- \sigma_a(\lambda, \x) L(\lxo) + \sigma_a(\lambda, \x) L_e(\lxo) \\
&- \sigma_s(\lambda, \x) L(\lxo) + \sigma_s(\lambda, \x) L_i(\lxo),
\end{split}
\end{equation}

where $L_i$ is defined by

\begin{equation}
L_i(\lxo) = \int_{4 \pi} L(\lxo_i) \Phi (\lambda, \omegam, \omegam_i) d \omegam_i,
\end{equation}

where $\sigma_a$ is an absorption coefficient, $\sigma_s$ is a scattering coefficient, $L_e$ is the emitted spectral radiance at the point, $L_i$ is the in-scattering radiance, $\Phi$ is a scattering phase function and $\omega_i$ is a scattering sampling direction.
In order to get an analytical solution to the aforementioned equation, the properties of the medium are assumed to be homogeneous over a small segment $\deltax$ in space,

\begin{equation}
\label{eq:rte_solution_paper}
\begin{split}
L(\lambda, \x + \Delta\x, \omegam) &= e^{-\sigma_t(\lambda, \x) \deltax} L(\lxo) +  \\
& (1 - e^{-\sigma_t(\lambda, \x) \deltax} ) \frac{\sigma_a(\lambda, \x) L_e(\lxo) + \sigma_s(\lambda, \x) L_i(\lxo)}{\sigma_t(\lambda, \x)},
\end{split}
\end{equation}

where $\sigma_t = \sigma_a + \sigma_s$ is the extinction coefficient.

\section{Scattering}
\label{sec:scattering}

The scattering phase function is defined by

\begin{equation}
\Phi (\lambda, \omegam, \omegam_i) = \frac{1 - g(\lambda)^2}{4 \pi(1 + g(\lambda)^2 - 2 g(\lambda) \omegam \omegam_i)^{\frac{3}{2}}},
\end{equation}

where $g$ can be a function of the wavelength, although in most cases is chosen to be constant.
The value of $g$ must be in the range $(-1, 1)$, where $g < 0$ corresponds to backwards scattering, $g = 0$ to isotropic scattering, and $g > 0$ to forwards scattering.

\section{Soot Absorption}
\label{sec:soot_absorption}

The spectral absorption coefficient of soot is defined as

\begin{equation}
\sigma_a(\lambda, \x) = \frac{48 N(\x) \pi R^3 n m}{\lambda^{\alpha(\lambda)} ((n^2 - m^2 +2)^2 + 4 n^2 m^2)},
\end{equation}

where $N(\x)$ is the number density, density per unit volume, $R$ is the radius of a soot particle, $n$, $m$ and $\alpha(\lambda) = 1.39$ are optical constants for different types of soot.
In Table~\ref{tb:soot_absorption_coefficients}, values for the optical constants $n$ and $m$ are provided for several materials, the data was obtained from~\cite{Dalzell:1969}.
The radius of soot particles was determined in the range $R \in \lbrace 50\mbox{~\AA} \ldots 800\mbox{~\AA} \rbrace $ by~\cite{Dalzell:1969}.
Since our data is defined with soot densities, we have chosen the radius to be the mean value $R = 425 \times 10^{-10}$ metres.

\begin{table}[htbp!]
\centering
\caption{Absorption constants for propane and acetylene, wavelengths are in nanometres.}
\label{tb:soot_absorption_coefficients}
\begin{tabular}{cc|c|c|c|c|}
\cline{3-6}
                                                 &    & \multicolumn{4}{c|}{Wavelengths} \\ \cline{3-6} 
                                                 &    & 435.8   & 450    & 550   & 650   \\ \hhline{--|=|=|=|=|}
\multicolumn{1}{|c|}{\multirow{2}{*}{Propane}}   & \multicolumn{1}{c||}{n}  & 1.57    & 1.56   & 1.57  & 1.56  \\ \cline{2-6} 
\multicolumn{1}{|c|}{}                           & \multicolumn{1}{c||}{nm} & 0.46    & 0.5    & 0.53  & 0.52  \\ \hline
\multicolumn{1}{|c|}{\multirow{2}{*}{Acetylene}} & \multicolumn{1}{c||}{n}  & 1.56    & 1.56   & 1.56  & 1.57  \\ \cline{2-6} 
\multicolumn{1}{|c|}{}                           & \multicolumn{1}{c||}{nm} & 0.46    & 0.48   & 0.46  & 0.44  \\ \hline
\end{tabular}
\end{table}

\section{Black Body Radiation}
\label{sec:black_body_radiation}

Planck's equation for black body radiation characterizes the electromagnetic radiation emitted by a black body in thermal equilibrium at a definite temperature $T$, is defined by

\begin{equation}
B_\lambda(T, \lambda, \eta) = \frac{2 h c^2}{\lambda^5  (e ^\frac{h c}{\lambda k T} - 1)},
\end{equation}

where $h = 6.62606957 \times 10^{-34}~J/s$ is Planck's constant, $k = 1.3806488 \times 10^{-23}~J/K$ is Boltzmann constant and $c =  c_0 / \eta$ is the speed of light in the current medium, where $c_0 = 299792458~m/s$ is the speed of light in a vacuum and $\eta$ is the refraction index of the medium.
The refraction index $\eta$ varies across the medium, the procedure to compute will be described in Section~\ref{sec:refraction}.

\section{Emission From Chemical Species}
\label{sec:emission_from_chemical_species}

The emission and absorption coefficients associated with a given spectral frequency can be computed as


\begin{align}
\sigma_a &= \frac{\phi(\lambda) N_2 A_{21} \lambda^4 (e ^\frac{h c}{\lambda k T} - 1)}{8 \pi c}, \\
j_\lambda &= \sigma_a B_\lambda (T, \lambda, n), 
\end{align}

where $\phi(\lambda)$ is the normalized spectral line, $N_2$ is the number density of elements, $A_{21}$ is an Einstein coefficient measuring the transition probabilities of an spontaneous emission, and $j_\lambda$ is the emitted radiance for the current chemical component.

\section{Refraction}
\label{sec:refraction}

Assuming a negligible reflection index, the refraction angles for the rays can be easily computed using Snell's law

\begin{equation}
\frac{\sin \theta_1}{\sin \theta_2} = \frac{\eta_2}{\eta_1},
\end{equation}

where $\theta_1$ is the incident angle, $\theta_2$ is the refracted angle, $\eta_1$ is the index of refraction of the media which the ray is coming from and $\eta_2$ is the index of refraction of the media which the ray is going to.

\cite{Ciddor:1996} proposed a method to compute the refractive indices of air 

\begin{equation}
\label{eq:ciddor_eta_air}
\eta_{air} = 1 + \frac{\rho_a \eta_{axs}}{\rho_{axs}} + \frac{\rho_w \eta_{ws}}{\rho_{ws}},
\end{equation}

where $\rho_{axs}$ is the density of dry air, $\rho_{ws} = 0.00985938$ is the density of pure water vapour, $\rho_{a}$ and $\rho_{w}$ are the equivalent quantities for dry air, $\eta_{axs}$ is the refractive index for $CO_2$ and $\eta_{ws}$ is the refractive index for water vapour. 

\begin{align}
\label{eq:ciddor_eta_axs}
\eta_{axs} &= \eta_{as} \left(1 + 0.534 \times 10^{-6} \left(x_c - 450 \right) \right), \\
\label{eq:ciddor_eta_as}
\eta_{as} &= 10^{-8} \left( \frac{k_1}{k_0 - \lambda} + \frac{k_3}{k_2 - \lambda} \right),
\end{align}

where $x_c \in \lbrace 300 \ldots 450 \rbrace $ is the proportion in ppm of $CO_2$, in our case $x_c = 450$, $k_0 = 238.0185~\mu m^{-2}$, $k_1 = 5792105~\mu m^{-2}$, $k_2 = 57.362~\mu m^{-2}$, $k_3 = 167917~\mu m^{-2}$ and $\lambda$ is the wave number (reciprocal of the vacuum wavelength).

\begin{equation}
\label{eq:ciddor_eta_ws}
\eta_{ws} = 1.022 \times 10^{-8} \left( w_0 + w_1 \lambda + w_1 \lambda^2 + w_3 \lambda^3 \right),
\end{equation}

where $w_0 = 295.235~\mu m^{-2}$, $w_1 = 2.6422~\mu m^{-2}$, $w_2 = -0.03238~\mu m^{-2}$ and $w_3 = 0.004028~\mu m^{-2}$

The $\rho$ densities are computed as follows

\begin{equation}
\label{eq:ciddor_rho}
\rho =  \frac{p m_a}{zrT} \left( 1 - x_w \left(1 - \frac{m_w}{m_a} \right) \right), 
\end{equation}

where $p$ is the pressure in pascals, $1~atm~ = 101 325~Pa$ in our case, $r = 8.31451~J mol^{-1} K^{-1}$, $m_w = 0.018015~kg/mol$, 

\begin{align}
\label{eq:ciddor_m_a}
m_a &= 0.0289635 + 12.011 \times 10^{-8}(x_c - 400)),\\
\label{eq:ciddor_z}
z &= 1 - \frac{p}{T} \left(a_0 + a_1 t + a_2 t^2 + \left(b_0 + b_1 t \right) x_w + \left(c_0 + c_1 t \right) x_w^2 \right) + \left( \frac{p}{T} \right)^2 \left( d + ex_w^2 \right),
\end{align}

where $a_0 = 1.58123~\times 10 ^{-6} K Pa^{-1}$,  $a_1 =-2.9331 ~\times 10 ^{-8} Pa^{-1}$,  $b_0 = 5.707~\times 10 ^{-6} K Pa^{-1}$,  $b_1 = -2.051~\times 10 ^{-8} Pa^{-1}$,  $c_0 = 1.9898~\times 10 ^{-4} K Pa^{-1}$,  $c_1 = -2.376~\times 10 ^{-6} K Pa^{-1}$,  $d = 1.83~\times 10 ^{-11} K^2 Pa^{-2}$ and  $e = -0.765~\times 10 ^{-8} K^2 Pa^{-2}$.

\begin{align}
\label{eq:ciddor_t}
T &= t + 273.15, \\
\label{eq:ciddor_x_w}
x_w &= \frac{f h v}{ p}, \\
\label{eq:ciddor_f}
f &= \alpha + \beta p + \gamma t^2,
\end{align}

where $\alpha = 1.00062$, $\beta = 3.14 \times 10^{-8} Pa^{-1}$ and $\gamma = 5.6 \times 10^{-7} \C^{-2}$, and $h \in \lbrace 0 \ldots 1 \rbrace$ is the air relative fractional humidity, $0$ for dry air and given by the user for moist air.

The pressure $p$ can also be computed using the ideal gas law
\begin{equation}
\label{eq:ciddor_p}
pV=\frac{NRT}{V} = n T,
\end{equation}

where $N$ is the number of molecules, $V$ is the total volume of the gas, $T$ is the temperature of the gas, $n$ is the number density, $R = k N_a$, where $k = 1.3806488 \times 10^{-23}~J/K$ is the Boltzmann constant and $N_a = 6.02214129 \times 10^{23}~mol^{-1}$ is Avogadro constant.

\subsection{Davis' Saturation Vapour Pressure}
\label{subsec:davis_v}

In \cite{Ciddor:1996} paper, the author used the method proposed by~\cite{Davis:1992} to compute the saturation of vapour pressure

\begin{equation}
\label{eq:davis_v}
v = e^{AT^2 + BT + C + D/T},
\end{equation}

where $A = 1.2378847 \times 10^{-5} K^{-2}$, $B = -1.9121316 \times 10^{-2} K^{-1}$, $C = 33.93711047$ and $D = -6.3431645 \times 10^3 K$.
This equation is relatively easy to compute, however incorrect results will be obtained for temperatures below $0\C$.
As we are concerned with fire rendering, which entails high temperatures, this method was chosen for its simplicity.

\subsection{Huang's Saturation Vapour Pressure}
\label{subsec:huang_v}

The International Association for the Properties of Water and Steam (IAPWS) adopted an alternative technique to~\cite{Davis:1992}, which was proposed by~\cite{Huang:1998}.
The more recent method will address the drawbacks of the previous one, giving reasonable results for temperatures below $0\C$, nevertheless, the generality comes with increased complexity in the equations.

\begin{align}
\label{eq:huang_v}
v &= 10^{6} \left( \frac{2C}{X} \right)^4, \\
X &= -B + \left( B^2 - 4AC \right)^{\frac{1}{2}}, \\
\begin{bmatrix}
A \\
B \\
C
\end{bmatrix} &=
\begin{bmatrix}
1 & K_1 & K_2 \\
K_3 & K_4 & K_5 \\
K_6 & K_7 & K_8 \\
\end{bmatrix} 
\begin{bmatrix}
~\Omega^2 \\
\Omega \\
1
\end{bmatrix}, \\
\Omega &= T + \frac{K_9}{T - K_{10}},
\end{align}

where $K_1 = 1.16705214528 \times 10^{3}$, $K_2 = -7.24213167032 \times 10^{5}$, $K_3 = -1.70738469401 \times 10$, $K_4 = 1.20208247025 \times 10^{4}$, $K_5 = -3.23255503223 \times 10^{6}$, $K_6 = 1.49151086135 \times 10$, $K_7 = -4.82326573616 \times 10^{3}$, $K_8 = 4.05113405421 \times 10^{5}$, $K_9 = -2.38555575678 \times 10^{-1}$, and $K_{10} = 6.50175348448 \times 10^{2}$.

\subsection{Ciddor's Method Summary}

The inputs of Ciddor's technique are a wavelength $\lambda_0$, a temperature $t$, a pressure $p$, an relative humidity value $h$ and a $CO_2$ concentration $x_c$, the steps to compute the air index of refraction are,

\begin{enumerate}
\item Precompute $z_a = 0.9995922115$, from Equation~\ref{eq:ciddor_z} with $t = 20~\C$, $p = 101.325~Pa$ and $x_w=0$.
\item Find $\lambda = 1 / \lambda_0^2$, and $T$ with Equation~\ref{eq:ciddor_t}.
\item Find $v$ with the method described in Section~\ref{subsec:davis_v} or in Section~\ref{subsec:huang_v}.
\item Find $x_w$ using Equations~\ref{eq:ciddor_f} and~\ref{eq:ciddor_x_w}.
\item Find $\eta_{as}$ using Equation~\ref{eq:ciddor_eta_as}.
\item Find $\eta_{ws}$ using Equation~\ref{eq:ciddor_eta_ws}.
\item Find $m_a$ using Equation~\ref{eq:ciddor_m_a}.
\item Find $\eta_{axs}$ using Equation~\ref{eq:ciddor_eta_axs}.
\item Find $z_m$ using Equation~\ref{eq:ciddor_z}.
\item Find $\rho_{axs} = (p_0 m_a)/(z_a r T_0)$, where $p_0 = 101325$, and $T_0 = 288.15$, using Equation~\ref{eq:ciddor_rho}.
\item Find $\rho_{w} = (x_w p m_w)/(z_m r T)$ using Equation~\ref{eq:ciddor_rho}.
\item Find $\rho_{a} = ((1 - x_w) p m_a)/(z_m r T)$ using Equation~\ref{eq:ciddor_rho}.
\item Find the index of refraction $\eta_{air}$ using Equation~\ref{eq:ciddor_eta_air}.
\end{enumerate} 

\section{Visual Adaptation}
\label{sec:visual_adaptation}

The human eye presents a non-linear response to incident radiance $L$, the reaction has been modelled  with certain success with the simple function 

\begin{equation}
R(L, \tau) = \frac{L}{L + \tau},
\end{equation}

where $\tau$ is a non-linear adaptation state.
This state is determined by the visual system to maximize the perception of features for a given scene.
