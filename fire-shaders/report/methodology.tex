%------------------------------------------------------------------------------
\chapter{Methodology}
\label{ch:methodology}

Not only explain in more detail every equation, what it means, comparison with other papers and ideally improvements or were it fails at least.
Should I explain what is and how to do ray marching? What is the spectrum? How to integrate it to RGB coefficients?

\section{Radiative Transport Equation}
\label{sec:radiative_transport_equation}

The Radiative Transport Equation (RTE) models the variation of spectral radiance in the medium $L(\lxo)$, where $\lambda$ is a given wavelength in $m$, $\x$ is the point of interest in space, and $\omegam$ is a vector that points towards the viewing direction.
The RTE is defined as

\begin{equation}
\begin{split}
(\omega \nabla) L(\lxo) = &- \sigma_a(\lambda, \x) L(\lxo) + \sigma_a(\lambda, \x) L_e(\lxo) \\
&- \sigma_s(\lambda, \x) L(\lxo) + \sigma_s(\lambda, \x) L_i(\lxo),
\end{split}
\end{equation}

where $L_i$ is defined by

\begin{equation}
L_i(\lxo) = \int_{4 \pi} L(\lxo_i) \Phi (\lambda, \omegam, \omegam_i) d \omegam_i,
\end{equation}

where $\sigma_a$ is an absorption coefficient, $\sigma_s$ is a scattering coefficient, $L_e$ is the emitted spectral radiance at the point, $L_i$ is the in-scattering radiance, $\Phi$ is a scattering phase function and $\omega_i$ is a scattering sampling direction.
In order to get an analytical solution to the aforementioned equation, the properties of the medium are assumed to be homogeneous over a small segment $\deltax$ in space,

\begin{equation}
\begin{split}
L(\lambda, \x + \Delta\x, \omegam) &= e^{-\sigma_t(\lambda, \x) \deltax} L(\lxo) +  \\
& (1 - e^{-\sigma_t(\lambda, \x) \deltax} ) \frac{\sigma_a(\lambda, \x) L_e(\lxo) + \sigma_s(\lambda, \x) L_i(\lxo)}{\sigma_t(\lambda, \x)},
\end{split}
\end{equation}

where $\sigma_t = \sigma_a + \sigma_s$ is the extinction coefficient.

\section{Scattering}
\label{sec:scattering}

The scattering phase function is defined by

\begin{equation}
\Phi (\lambda, \omegam, \omegam_i) = \frac{1 - g(\lambda)^2}{4 \pi(1 + g(\lambda)^2 - 2 g(\lambda) \omegam \omegam_i)^{\frac{3}{2}}},
\end{equation}

where $g$ can be a function of the wavelength, although in most cases is chosen to be constant.
The value of $g$ must be in the range $(-1, 1)$, where $g < 0$ corresponds to backwards scattering, $g = 0$ to isotropic scattering, and $g > 0$ to forwards scattering.

\section{Soot Absorption}
\label{sec:soot_absorption}

The spectral absorption coefficient of soot is defined as

\begin{equation}
\sigma_a(\lambda, \x) = \frac{48 N(\x) \pi R^3 n m}{\lambda^{\alpha(\lambda)} ((n^2 - m^2 +2)^2 + 4 n^2 m^2)},
\end{equation}

where $N(\x)$ is the number density, density per unit volume, $R$ is the radius of a soot particle, $n$, $m$ and $\alpha(\lambda)$ are optical constants for different types of soot.

\section{Black Body Radiation}
\label{sec:black_body_radiation}

Planck's equation for black body radiation characterizes the electromagnetic radiation emitted by a black body in thermal equilibrium at a definite temperature $T$, is defined by

\begin{equation}
B_\lambda(T, \lambda, n) = \frac{2 h c^2}{\lambda^5  (e ^\frac{h c}{\lambda k T} - 1)},
\end{equation}

where $h$ is Planck's constant, $k$ is Boltzmann constant and $c = \frac{c_0}{n_r}$ is the speed of light in the current medium, where $c_0$ is the speed of light in a vacuum and $n_r$ is the refraction index of the medium.

\section{Emission From Chemical Species}
\label{sec:emission_from_chemical_species}



\section{Refraction}
\label{sec:refraction}

Add Ciddor's equation

Assuming a negligible reflection index, the refraction angles for the rays can be easily computed using Snell's law

\begin{equation}
\frac{\sin \theta_1}{\sin \theta_2} = \frac{\eta_2}{\eta_1},
\end{equation}

where $\theta_1$ is the incident angle, $\theta_2$ is the refracted angle, $\eta_1$ is the index of refraction of the media which the ray is coming from and $\eta_2$ is the index of refraction of the media which the ray is going to.

\section{Visual Adaptation}
\label{sec:visual_adaptation}

The human eye presents a non-linear response to incident radiance $L$, the reaction has been modelled  with certain success with the simple function 

\begin{equation}
R(L, \tau) = \frac{L}{L + \tau},
\end{equation}

where $\tau$ is a non-linear adaptation state.
This state is determined by the visual system to maximize the perception of features for a given scene.
