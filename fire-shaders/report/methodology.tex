%------------------------------------------------------------------------------
\chapter{Methodology}
\label{ch:methodology}

Not only explain in more detail every equation, what it means, comparison with other papers and ideally improvements or were it fails at least.
Should I explain what is and how to do ray marching? What is the spectrum? How to integrate it to RGB coefficients?
Add all the values for the constants here or in implementation details??

\section{Radiative Transport Equation}
\label{sec:radiative_transport_equation}

The Radiative Transport Equation (RTE) models the variation of spectral radiance in the medium $L(\lxo)$, where $\lambda$ is a given wavelength in $m$, $\x$ is the point of interest in space, and $\omegam$ is a vector that points towards the viewing direction.
The RTE is defined as

\begin{equation}
\begin{split}
(\omega \nabla) L(\lxo) = &- \sigma_a(\lambda, \x) L(\lxo) + \sigma_a(\lambda, \x) L_e(\lxo) \\
&- \sigma_s(\lambda, \x) L(\lxo) + \sigma_s(\lambda, \x) L_i(\lxo),
\end{split}
\end{equation}

where $L_i$ is defined by

\begin{equation}
L_i(\lxo) = \int_{4 \pi} L(\lxo_i) \Phi (\lambda, \omegam, \omegam_i) d \omegam_i,
\end{equation}

where $\sigma_a$ is an absorption coefficient, $\sigma_s$ is a scattering coefficient, $L_e$ is the emitted spectral radiance at the point, $L_i$ is the in-scattering radiance, $\Phi$ is a scattering phase function and $\omega_i$ is a scattering sampling direction.
In order to get an analytical solution to the aforementioned equation, the properties of the medium are assumed to be homogeneous over a small segment $\deltax$ in space,

\begin{equation}
\label{eq:rte_solution_paper}
\begin{split}
L(\lambda, \x + \Delta\x, \omegam) &= e^{-\sigma_t(\lambda, \x) \deltax} L(\lxo) +  \\
& (1 - e^{-\sigma_t(\lambda, \x) \deltax} ) \frac{\sigma_a(\lambda, \x) L_e(\lxo) + \sigma_s(\lambda, \x) L_i(\lxo)}{\sigma_t(\lambda, \x)},
\end{split}
\end{equation}

where $\sigma_t = \sigma_a + \sigma_s$ is the extinction coefficient.

\section{Scattering}
\label{sec:scattering}

The scattering phase function is defined by

\begin{equation}
\Phi (\lambda, \omegam, \omegam_i) = \frac{1 - g(\lambda)^2}{4 \pi(1 + g(\lambda)^2 - 2 g(\lambda) \omegam \omegam_i)^{\frac{3}{2}}},
\end{equation}

where $g$ can be a function of the wavelength, although in most cases is chosen to be constant.
The value of $g$ must be in the range $(-1, 1)$, where $g < 0$ corresponds to backwards scattering, $g = 0$ to isotropic scattering, and $g > 0$ to forwards scattering.

\section{Soot Absorption}
\label{sec:soot_absorption}

The spectral absorption coefficient of soot is defined as

\begin{equation}
\sigma_a(\lambda, \x) = \frac{48 N(\x) \pi R^3 n m}{\lambda^{\alpha(\lambda)} ((n^2 - m^2 +2)^2 + 4 n^2 m^2)},
\end{equation}

where $N(\x)$ is the number density, density per unit volume, $R$ is the radius of a soot particle, $n$, $m$ and $\alpha(\lambda)$ are optical constants for different types of soot.

\section{Black Body Radiation}
\label{sec:black_body_radiation}

Planck's equation for black body radiation characterizes the electromagnetic radiation emitted by a black body in thermal equilibrium at a definite temperature $T$, is defined by

\begin{equation}
B_\lambda(T, \lambda, n) = \frac{2 h c^2}{\lambda^5  (e ^\frac{h c}{\lambda k T} - 1)},
\end{equation}

where $h$ is Planck's constant, $k$ is Boltzmann constant and $c =  c_0 / n_r$ is the speed of light in the current medium, where $c_0$ is the speed of light in a vacuum and $n_r$ is the refraction index of the medium.

\section{Emission From Chemical Species}
\label{sec:emission_from_chemical_species}

The emission and absorption coefficients associated with a given spectral frequency can be computed as


\begin{align}
\sigma_a &= \frac{\phi(\lambda) N_2 A_{21} \lambda^4 (e ^\frac{h c}{\lambda k T} - 1)}{8 \pi c}, \\
j_\lambda &= \sigma_a B_\lambda (T, \lambda, n), 
\end{align}

where $\phi(\lambda)$ is the normalized spectral line, $N_2$ is the number density of elements, $A_{21}$ are Einstein coefficients measuring the transition probabilities of spontaneous emission, and $j_\lambda$ is the emitted radiance for the current chemical component.

\section{Refraction}
\label{sec:refraction}

Add Ciddor's equation

REAL CITE Cidor proposed a method to compute the refractive indices of air 

\begin{equation}
n_{air} = 1 + \frac{\rho_a( n_{axs} - 1)}{\rho_{axs}} + \frac{\rho_w( n_{ws} - 1)}{\rho_{ws}},
\end{equation}

where $\rho_{axs}$ is the density of dry air, $\rho_{ws}$ is the density of pure water vapour, $\rho_{a}$ and $\rho_{w}$ are the equivalent quantities for dry air, $n_{axs}$ is the refractive index for $CO_2$ and $n_{ws}$ is the refractive index for water vapour. 

\begin{align}
n_{axs} &= 1 + (n_{as} - 1)(1 + 0.534 \times 10^{-6} (x_c - 450)), \\
n_{as} &= 10^{-8} \left( 1 + \frac{k_1}{k_0 - \lambda^2} + \frac{k_3}{k_2 - \lambda^2} \right),
\end{align}

where $x_c \in \lbrace 300 \ldots 450 \rbrace $ is the proportion in ppm of $CO_2$, in our case $x_c = 450$, $k_0 = 238.0185~\mu m^{-2}$, $k_1 = 5792105~\mu m^{-2}$, $k_2 = 57.362~\mu m^{-2}$, $k_3 = 167917~\mu m^{-2}$ and $\lambda$ is the wave number (reciprocal of the vacuum wavelength).

\begin{equation}
n_{ws} = 10^{-8} \left( 1 + cf(w_0 + w_1 \lambda^2 + w_1 \lambda^4 + w_3 \lambda^6) \right),
\end{equation}

where $cf = 1.022$, $w_0 = 295.235~\mu m^{-2}$, $w_1 = 2.6422~\mu m^{-2}$, $w_2 = -0.03238~\mu m^{-2}$ and $w_3 = 0.004028~\mu m^{-2}$

The $\rho$ densities are computed as follows

\begin{equation}
\rho =  \frac{p m_a}{zrT} \left( 1 - x_w \left(1 - \frac{m_w}{m_a} \right) \right), 
\end{equation}

where $p$ is the pressure in pascals, $1~atm$ in our case, $m_a = 10^{-3}(28.9635 + 12.011 \times 10^{-6}(x_c - 400))$ molar mass of dry air, $r = 8.31451~J mol^{-1} K^{-1}$, $m_w = 0.018015~kg/mol$, 

\begin{equation}
z = 1 - \frac{p}{T} \left(a_0 + a_1 t + a_2 t^2 + \left(b_0 + b_1 t \right) x_w + \left(c_0 + c_1 t \right) x_w^2 \right) + \left( \frac{p}{T} \right)^2 \left( d + ex_w^2 \right),
\end{equation}

where $a_0 = 1.58123~\times 10 ^{-6} K Pa^{-1}$,  $a_1 =-2.9331 ~\times 10 ^{-8} Pa^{-1}$,  $b_0 = 5.707~\times 10 ^{-6} K Pa^{-1}$,  $b_1 = -2.051~\times 10 ^{-8} Pa^{-1}$,  $c_0 = 1.9898~\times 10 ^{-4} K Pa^{-1}$,  $c_1 = -2.376~\times 10 ^{-6} K Pa^{-1}$,  $d = 1.83~\times 10 ^{-11} K^2 Pa^{-2}$ and  $e = -0.765~\times 10 ^{-8} K^2 Pa^{-2}$.

$x_w = \frac{f h v}{ p}$

$f = \alpha + \beta p + \gamma t^2$,

where $\alpha = 1.00062$, $\beta = 3.14 \times 10^{-8} Pa^{-1}$, $\gamma = 5.6 \times 10^{-7} C^{-2}$ and $t = T - 273.15$.

$v = e^{AT^2 + BT + C + D/T}$,

where $A = 1.2378847 \times 10^{-5} K^{-2}$, $B = -1.9121316 \times 10^{-2} K^{-1}$, $C = 33.93711047$ and $D = -6.3431645 \times 10^3 K$, and $h \in \lbrace 0 \ldots 1 \rbrace$ is the fractional humidity, $0$ for dry air and ??? for moist air.

The pressure $p$ can be computed using Boyle law of ideal gasses ...

Write it ?????

Assuming a negligible reflection index, the refraction angles for the rays can be easily computed using Snell's law

\begin{equation}
\frac{\sin \theta_1}{\sin \theta_2} = \frac{\eta_2}{\eta_1},
\end{equation}

where $\theta_1$ is the incident angle, $\theta_2$ is the refracted angle, $\eta_1$ is the index of refraction of the media which the ray is coming from and $\eta_2$ is the index of refraction of the media which the ray is going to.

\section{Visual Adaptation}
\label{sec:visual_adaptation}

The human eye presents a non-linear response to incident radiance $L$, the reaction has been modelled  with certain success with the simple function 

\begin{equation}
R(L, \tau) = \frac{L}{L + \tau},
\end{equation}

where $\tau$ is a non-linear adaptation state.
This state is determined by the visual system to maximize the perception of features for a given scene.
