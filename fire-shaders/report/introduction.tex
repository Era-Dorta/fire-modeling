%------------------------------------------------------------------------------
\chapter{Introduction}
\label{ch:introduction}
\textquote{\textit{A case that can be made for fire being, next to the life process, the most
complex of phenomena to understand}.} - Hoyt Hottle

For centuries humans has been attracted to fire due to its attractive presence and its dangerous nature.
Understanding and simulating combustion phenomena has many applications, such as in the film and computer games industries, where it is widely use in visual effects; or in the engineering community, where the modelling of combustion in engines or fire safety evaluations are frequently demanded.
In these and in many other applications, using real flames is an expensive and hazardous endeavour.

Fire can be modelled as a fluid, however its behaviour more complex, due to its multiphase flow, chemical reactions and radiative heat transport, than other fluids, such as water or smoke, which the computer graphics community have intensively researched.
As a result of the aforementioned complexity and the interdisciplinary nature of the problem, fire simulation is still an open problem in computer graphics.

A great deal of work done in the area has sacrificed complexity for interactiveness, therefore producing simplified models which hope to deceit the observer by exploiting the chaotic essence in fire motion.
Nevertheless, physically-based simulations incorporate the intrinsic processes that occur in a combustion scenario:

\begin{description}
\item[Flame motion:]
\item[Fuel erosion:] when the fuel reaches a certain temperature, it is vaporized into a gaseous state, which rises under the influence of buoyancy.
\item[Black body radiation:] the chemical species present in the fuel and the byproducts of the combustions emit energy in various wavelengths.
\end{description}

In order to be able to produce a realistic result, all of the preceding characteristics have to be taken into consideration.  